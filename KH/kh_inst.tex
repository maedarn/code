\documentclass[11pt,a4paper]{jsarticle}

\usepackage{amsmath,amssymb}
\usepackage{bm}
\usepackage{graphicx}
\usepackage{ascmac}
\usepackage{fancyhdr}

\pagestyle{fancy}
\lhead{Kelvin Helmholtz instability 担当:冨永}



\setlength{\textwidth}{\fullwidth}
\setlength{\textheight}{39\baselineskip}
\addtolength{\textheight}{\topskip}
\setlength{\voffset}{-0.5in}
\setlength{\headsep}{0.3in}

\def\vec#1{\mbox{\boldmath $#1$}}

\begin{document}

\section*{\S29 Instability of tangential discontinuities}
$\S$29の目標:非圧縮流体の接線不連続面の不安定性(Kelvin Helmholtz不安定性)の理解
\\

密度の不連続面内に$xy$平面をとり、面の法線方向を$z$軸にとる。この不連続面($z=0$)は接線不連続面\footnote{法線方向の速度が0で圧力は連続的に分布する面。}であるとし、不連続面をまたいで$x$方向の速度差が存在するとする。ここでは$z>0$(領域2)で速度$v_x=v$、$z<0$(領域2)で速度$v_x=0$とする。以下ではそれぞれの領域の密度と圧力には添え字1もしくは2をつける。

不連続面に摂動を与える(図は板書)。まず領域1(非摂動状態の速度が$v$)における摂動量が満たす関係式を導く。密度、圧力、速度の摂動を$\rho',\;p',\;\vec{v}'$とすると、連続の式とEulerの式から
\begin{gather}
\vec{\mathrm{div}}\vec{v}'=0,\\
\frac{\partial \vec{v}'}{\partial t}+\vec{v}\cdot\vec{\mathrm{grad}}\vec{v}'=-\frac{\vec{\mathrm{grad}}p'}{\rho},
\end{gather}
を得る。非摂動状態の速度$\vec{v}$は$x$成分しかないので、Euler方程式は
\begin{equation}
\frac{\partial \vec{v}'}{\partial t}+v\frac{\partial\vec{v}'}{\partial x}=-\frac{\vec{\mathrm{grad}}p'}{\rho}\tag{29.1}
\end{equation}
と変形できる。この式の両辺で$\vec{\mathrm{div}}$をとると、非圧縮性からLaplace方程式
\begin{equation}
\Delta p'=0\tag{29.2}
\end{equation}
を得る。


摂動を与えた後の不連続面を$z=\zeta(x,t)$と表し、不連続面の$z$軸方向の速度摂動を以下のように近似する\footnote{厳密には、流体の速度は流体素片に沿って位置を時間微分したもの。ここで言っている“近似"は、“不連続面を構成する流体素片はほとんど$z$方向にしか運動しない”ということを意味している(たぶん)。つまり位置$x$が不連続面内の流体素片のラベルになっている。}
\begin{equation}
v_z'=\frac{\mathrm{d}\zeta}{\mathrm{d}t}=\frac{\partial \zeta}{\partial t}+v\frac{\partial \zeta}{\partial x}.
\end{equation}
書き換えると式(29.3)に一致
\begin{equation}
\frac{\partial \zeta}{\partial t}=v_z'-v\frac{\partial \zeta}{\partial x}.\tag{29.3}
\end{equation}
上記の方程式系を用いて$p'=f(z)\exp{[i(kx-\omega t)]}$と書ける解を探す\footnote{$x$軸方向の周期性は摂動の入れ方から要請(たぶん)。$f(z)$は非圧縮性を反映する部分。}。式(29.2)より、
\begin{equation}
\frac{\mathrm{d}^2f}{\mathrm{d}z^2}-k^2f=0.
\end{equation}
これを満たす関数は$f(z)=\mathrm{constant}\times \exp{[\pm kz]}$。まず$z>0$を考えると圧力摂動は$p'=\mathrm{constant}\times\exp{[i(kx-\omega t)-kz]}$となる\footnote{不連続面から遠く離れた場所では摂動の影響がないということを反映。}。$v_z'$も$\exp{[i(kx-\omega t)]}$に比例するとして式(29.1)の$z$成分を計算すると、
\begin{equation}
i(kv-\omega)v_z'=-(-k)\frac{p_1'}{\rho_1}=\frac{kp_1'}{\rho_1}
\end{equation}
となり、$v_z'$は
\begin{equation}
v_z'=\frac{kp_1'}{i\rho_1(kv-\omega)}.\tag{29.5}
\end{equation}
さらにここで$\zeta$も$\exp{[i(kx-\omega t)]}$に比例していると考えると、式(29.3)から$v_z'=i\zeta(kv-\omega)$を得る。よって式(29.5)から
\begin{equation}
p_1'=-\zeta\rho_1(kv-\omega)^2/k\tag{29.6}
\end{equation}
となり、$z>0$で圧力の摂動$p_1'$と不連続面の摂動$\zeta$の関係を得た。

$z<0$における$p_2'$と$\zeta$の関係は、式(4)の解$f(z)$が$f(z)=\mathrm{constant}\times\exp{[kz]}$となることと、非摂動状態の速度が0であることを考えると、式(29.5)(29.6)から
\begin{equation}
p_2'=\zeta\rho_2\omega^2/k\tag{29.7}
\end{equation}
となる。

接線不連続面で圧力平衡が成り立つので、$p_1'=p_2'$とすると、式(29.6)(29.7)から$\rho_1(kv-\omega)^2=-\rho_2\omega^2$となる。これを$\omega$について解くと、
\begin{equation}
\omega=kv\frac{\rho_1\pm i\sqrt{\rho_1\rho_2}}{\rho_1+\rho_2}\tag{29.8}
\end{equation}
となりKelvin Helmholtz不安定性の分散関係を得る。ここからわかることをまとめると以下のようになる
\begin{itemize}
\item 接線不連続面は、どの波長の摂動に対しても常に不安定(過安定),
\item 位相速度($\omega$の実部の方)は重心の速度になってる,
\item 無限小の(0でない)速度差が存在すれば必ず不安定が成長する,
\item $\rho_1=\rho_2$のときにも不安定性は起こる.
\end{itemize}

以上の解析は流体の粘性が非常に小さい時に成り立つものである。In that case, it is meaningless to distinguish convected and absolute instability, since as $k$ increases the imaginary part of $\omega$ increases without limit, and hence the amplification coefficient of the perturbation as it is carried along may be as large as we please.

有限の粘性が流体に働く場合、接線不連続面は有限の厚みをもつ。このような場合の不連続面の安定性問題は、(速度分布における屈折(inflexion)の観点で??)境界層内の層流の安定性問題と数学的に同じである。室内実験や数値実験によってこのような場合も接線不連続面は不安定であることが示唆されていて、Kelvin Helmholtz不安定性はおそらく常に存在すると言える。
\end{document}